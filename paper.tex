\documentclass[10pt,twocolumn]{article}

% PACKAGES for formatting and symbols
\usepackage[utf8]{inputenc}
\usepackage[margin=0.75in]{geometry}
\usepackage{amsmath}
\usepackage{amssymb}
\usepackage{graphicx}
\usepackage{subcaption}
\usepackage{hyperref}
\usepackage{enumitem}
\usepackage{multicol}
\usepackage{caption}
\usepackage{authblk}
\usepackage{times}
\usepackage{titlesec}

% Configure captions
\captionsetup{font=small,labelfont=bf}

% Configure section titles
\titleformat{\section}
  {\normalfont\fontsize{11}{13}\bfseries}{\thesection}{1em}{}
\titleformat{\subsection}
  {\normalfont\fontsize{10}{12}\bfseries}{\thesubsection}{1em}{}
\titleformat{\subsubsection}
  {\normalfont\fontsize{10}{12}\itshape}{\thesubsubsection}{1em}{}

% Configure hyperlinks
\hypersetup{
    colorlinks=true,
    linkcolor=blue,
    citecolor=blue,
    urlcolor=blue,
}

% --- DOCUMENT INFORMATION ---
\title{\LARGE\textbf{A Graph-Theoretic Analysis of Contagion Dynamics on Simulated Network Topologies}}

\author[1]{Rishi Ahuja}
\author[1]{Gaurav}
\author[1]{Mohit Kumar}
\author[1]{Anish Ranjan}
\author[1]{Priyansh Kumar}
\author[1]{Hiten Janjua}

\affil[1]{\small Department of Information Technology\\
\textit{Roll Numbers: 24124092, 24124034, 24124069, 24124015, 24124086, 24124041}}

\date{\today}

% --- BEGIN DOCUMENT ---
\begin{document}

\maketitle

\begin{abstract}
\noindent
The modern world is defined by cascades — the rapid, often unpredictable spread of phenomena through interconnected systems. From the global reach of the COVID-19 pandemic to the instantaneous propagation of digital information, the underlying network structure is paramount. This paper presents a computational investigation into how network topology dictates the dynamics of contagion. Our central thesis is that the architecture of a network, more than the intrinsic properties of the contagion itself, determines the scale, speed, and severity of a cascade.

To explore this, we employ a two-pronged approach rooted in discrete mathematics. First, using the principles of Graph Theory, we programmatically construct and analyze three canonical network topologies at scale: the uniform random (Erdős-Rényi) network as a null model, the highly clustered small-world (Watts-Strogatz) network that mimics real-world social circles, and the hub-dominated scale-free (Barabási-Albert) network characteristic of online platforms and biological systems. Second, we utilize Predicate Logic to formally define the state-transition rules of a Susceptible-Infected-Recovered (SIR) epidemiological model, ensuring a rigorous and unambiguous foundation for our simulations.

Our experimental results reveal starkly different cascade dynamics across these topologies. While random and small-world networks exhibit contained, predictable epidemic curves, the scale-free topology consistently produces explosive, system-wide cascades from a single initial seed. The findings highlight the critical role of high-degree ``hub'' nodes, which act as super-spreaders, dramatically lowering the epidemic threshold and accelerating the outbreak. We conclude that network structure is a dominant, predictable variable in contagion, with profound implications for creating robust, topology-aware intervention strategies in public health, cybersecurity, and information science.
\end{abstract}

\section{Introduction}

The 21st century is defined by networks. From the rapid, global spread of the COVID-19 pandemic to the instantaneous propagation of a viral meme on social media, we are witnessing the profound power of cascades—events that spread through an interconnected system. While the ``what'' of the cascade often gets the most attention (the virus, the idea, the market crash), the ``how'' is fundamentally governed by the hidden architecture of the network it travels upon. A pathogen that might be a minor local event in one community could become an unstoppable global crisis in another, not because the pathogen changed, but because the structure of the connections was different.

This paper presents a computational investigation into this phenomenon. Our central thesis is that the \textbf{topology} of a network is a dominant and predictable factor in the dynamics of contagion. We move beyond simple descriptions and instead use the rigorous language of discrete mathematics to model, simulate, and analyze these cascades. By constructing different network ``worlds'' based on formal mathematical rules and running controlled epidemic simulations within them, we aim to demonstrate how properties like ``hubs'' and ``clusters'' are not abstract concepts but are the very mechanisms that dictate the speed, scale, and severity of an outbreak. This exploration provides a framework for understanding not only epidemics but any process of diffusion, from financial contagions to the spread of innovation.

\section{The Mathematical Foundation: A Primer on Graph Theory}

To analyze networks, we must first learn their language. The field of discrete mathematics provides a powerful and elegant framework for this, known as \textbf{Graph Theory}. A graph is not a chart or a plot; rather, it is a formal representation of a set of objects and the relationships between them. This section will introduce the fundamental concepts required to understand the structure of any network.

\subsection{Nodes and Edges: The Building Blocks}

At its core, every graph consists of two simple elements:

\begin{itemize}[leftmargin=*]
    \item \textbf{Node (or Vertex):} A node is an individual entity or object within the network. In our epidemic simulation, each node will represent a single person. In a social network, a node is a user profile. On the internet, a node could be a website or a server. They are the fundamental ``things'' we are studying.
    
    \item \textbf{Edge (or Link):} An edge is a connection between two nodes. It represents a relationship or interaction. For our simulation, an edge between two ``person'' nodes means they are in sufficient contact to potentially transmit a virus. On Twitter, an edge could represent a ``follow'' relationship.
\end{itemize}

Together, a collection of nodes and the edges connecting them form a graph, providing a precise mathematical map of a network.

\subsection{Degree: A Node's Local Influence}

Not all nodes in a network are created equal. One of the simplest and most important ways to measure a node's influence is by its \textbf{degree}.

\begin{itemize}[leftmargin=*]
    \item \textbf{Degree:} The degree of a node is the total number of edges connected to it. In a social network, a person's degree is their number of friends or followers.
\end{itemize}

The concept of degree allows us to move from just looking at a network's map to quantifying the properties of its members. For example, a node with a very high degree (a ``hub'') has a direct connection to a large portion of the network. As we will see in our simulations, the existence and number of these high-degree hubs is one of the most critical factors in determining how quickly a cascade can spread. It's the difference between a local outbreak and an explosive pandemic.

\subsection{Path Length: The Degrees of Separation}

Beyond the local influence of a single node, we also need to understand how information or a contagion can travel across the entire network. The concept that measures this is \textbf{path length}.

\begin{itemize}[leftmargin=*]
    \item \textbf{Path:} A path is a sequence of edges that connects a sequence of distinct nodes. Think of it as a route from a starting point to a destination, moving from friend to friend.
    
    \item \textbf{Path Length:} The length of a path is the number of edges it contains. The shortest path between two nodes is the most efficient route a piece of information or a virus can take.
    
    \item \textbf{Average Path Length:} For an entire network, the average path length is the average of the shortest path lengths between all possible pairs of nodes. This metric gives us a single, powerful number to describe the overall connectivity of the network. A low average path length means the network is highly connected and ``small,'' suggesting that a cascade can spread from one side to the other with surprising speed. This is the mathematical idea behind the famous ``six degrees of separation'' concept.
\end{itemize}

A network with a short average path length is a fertile ground for rapid, widespread cascades. Information doesn't have to travel far to reach even the most remote corners of the graph.

\subsection{Clustering Coefficient: Measuring Network `Cliquey-ness'}

Finally, we need a way to measure the local structure and density of a network. While path length tells us about global connectivity, the clustering coefficient tells us about the tendency of nodes to form tight-knit local groups or ``cliques.''

\begin{itemize}[leftmargin=*]
    \item \textbf{Clustering Coefficient:} For a single node, its local clustering coefficient measures how connected its neighbors are to each other. It answers the question: ``What fraction of my friends are also friends with each other?'' A high clustering coefficient for a node means it belongs to a dense, well-connected group.
    
    \item \textbf{Average Clustering Coefficient:} The average clustering coefficient for the entire network is the average of the local coefficients of all its nodes.
\end{itemize}

A network with a high average clustering coefficient is characterized by many dense local pockets of connections. This has a fascinating effect on contagion. On one hand, the dense clusters can cause intense local outbreaks. On the other, the spread \emph{between} these clusters might be slow if there are few connecting ``bridge'' edges. As we will see, the interplay between a network's clustering and its path length is a key determinant of its overall dynamics.

\section{Modeling Society: Three Network Topologies}

Graph theory provides the tools to describe any network, but real-world networks are not all the same. The connections in a random group of strangers are vastly different from the connections between users on Twitter or the neurons in a brain. To investigate how these structural differences impact a cascade, we will construct and analyze three foundational network models. Each is generated by a distinct set of rules and exhibits unique properties that make it a useful approximation for different types of real-world systems.

\subsection{The Random Network: A World Without Structure (Erdős-Rényi Model)}

\subsubsection{Motivation and Historical Context}

Before we could understand the complex, structured networks of the real world, science first needed a way to understand the properties of a network formed by pure, unstructured chance. This was the intellectual ground zero for network theory, a question tackled in the 1950s by the brilliant and famously eccentric mathematician Paul Erdős and his collaborator Alfréd Rényi.

Their work was not an attempt to perfectly model a human social network; they knew real life was far more complex. Instead, their goal was to answer a set of fundamental mathematical questions. If you start with a set of isolated points and begin adding connections between them at random, what happens? At what precise moment does a connected web emerge from the isolated fragments? Do large, cohesive groups form, or does the network remain a fractured collection of small clusters? The \textbf{Erdős-Rényi (ER) model} was their framework for answering these questions.

In the context of our paper, the ER model serves as the essential \textbf{null hypothesis}. It is the baseline reality of a world with no social biases, no geographical constraints, and no preferential attachments. By first understanding how a cascade behaves in this sterile, randomized environment, we can later appreciate the profound impact that structure and order bring to the system. It is the scientific control group against which our other, more realistic models will be measured.

\subsubsection{Formal Construction and Parameters}

There are two closely related formal definitions of the ER model. We will focus on the one most commonly used in computational modeling, denoted as $G(N, p)$.

\begin{itemize}[leftmargin=*]
    \item \textbf{$N$}: The total number of nodes (vertices) in the graph. This is a fixed, predetermined number.
    \item \textbf{$p$}: The probability that an edge exists between any two distinct nodes. This value is constant for all pairs of nodes in the network.
\end{itemize}

The algorithm to generate a $G(N, p)$ graph is as follows:

\begin{enumerate}
    \item \textbf{Initialization}: Begin with a set of $N$ nodes, completely isolated from one another.
    
    \item \textbf{Enumeration of Pairs}: Identify all possible pairs of nodes. For $N$ nodes, the total number of unique pairs is given by the binomial coefficient $\binom{N}{2} = \frac{N(N-1)}{2}$. For even a moderately sized network of 5,000 nodes, this amounts to nearly 12.5 million potential connections.
    
    \item \textbf{Probabilistic Edge Creation}: For each of these potential connections, a random, independent trial is performed. A random number is generated, typically between 0 and 1. If this number is less than or equal to $p$, an edge is drawn between the two nodes. If the number is greater than $p$, no edge is drawn.
    
    \item \textbf{Finalization}: After all pairs have been considered, the resulting graph is a single instance of a $G(N, p)$ random network.
\end{enumerate}

It's crucial to understand that every time you run this algorithm, you will get a slightly different graph, but they will all share the same statistical properties dictated by $N$ and $p$. The parameter $p$ is the sole tuning knob; a small $p$ creates a sparse, fragmented graph, while a large $p$ creates a dense, highly connected one.

\subsubsection{Detailed Properties and Mathematical Analysis}

The beauty of the ER model is that its macroscopic properties can be precisely described by mathematics.

\paragraph{Degree Distribution.} In a random network, a node's final number of connections is the result of many small, independent chances. To build a strong intuition for the resulting pattern, let's use an analogy: imagine a large sidewalk divided into thousands of squares, and it begins to rain lightly. Each individual raindrop is a rare, independent event for any single square.

\begin{itemize}[leftmargin=*]
    \item \textbf{The Question:} After a minute, what will the pattern of raindrops look like? Will some squares be flooded while others are bone dry?
    
    \item \textbf{The Logic:} For any given square, the chance of being hit by any single raindrop is minuscule. Because of this, it's highly probable that a square will be missed by all the raindrops, ending up with \textbf{zero} hits. It's also reasonably likely that a square might get hit by \textbf{one} raindrop. It's much less likely it would be hit by \textbf{two}, and the probability of it being hit by ten or more is practically zero.
    
    \item \textbf{The Result:} If you were to count the number of raindrops in each square and plot the results, you would get the \textbf{Poisson distribution}. It would show a large number of squares with zero or one raindrop, and the counts would fall off dramatically for higher numbers.
\end{itemize}

This is precisely what happens in our $G(N, p)$ random network. Each node is a ``sidewalk square,'' and each of the $N-1$ other nodes is a potential ``raindrop.'' Since the probability $p$ of a connection is small, each potential link is a rare event. Therefore, the final distribution of degrees (connections) will follow this same Poisson pattern.

Most nodes will have a degree very close to the network's average ($\lambda = p(N-1)$). The probability of finding a node with a degree that is significantly higher than this average drops off exponentially. This isn't just an observation; it's a mathematical certainty. In a large random network, the rules of probability make the emergence of massive \textbf{hubs a statistical impossibility}. The network is structurally democratic; there's a ``typical'' node, and almost everyone is typical. This fundamental property is one of the model's most significant departures from many real-world networks.

\paragraph{Clustering Coefficient.} The ER model exhibits very low clustering. To understand why, consider one node, Alice, and two of her neighbors, Bob and Carol. For Alice's local clustering coefficient to be high, Bob and Carol must also be connected to each other. In a random graph, the existence of the Alice-Bob and Alice-Carol links has \textbf{no influence} on the probability of a Bob-Carol link. That probability remains simply $p$. For a large network, $p$ is typically very small, so the clustering coefficient, which is approximately equal to $p$, is also very small. The network is fundamentally non-local; friendships are not concentrated in ``cliques.''

\paragraph{Path Length and the Emergence of the Giant Component.} The most fascinating property of a random graph is its dramatic \textbf{phase transition}. It doesn't just grow bigger smoothly; it fundamentally changes its character at a critical tipping point. This transition is best understood by observing the size of the largest connected cluster of nodes—the ``giant component''—as we slowly increase the network's average degree, $\lambda$.

\begin{itemize}[leftmargin=*]
    \item \textbf{When $\lambda < 1$ (Subcritical Phase):} When the average node has less than one connection, the network is a fragmented archipelago of tiny, isolated islands. A contagion starting on one island cannot spread to another. The largest component is minuscule, containing only a logarithmic number of nodes ($\log(N)$).
    
    \item \textbf{When $\lambda = 1$ (Critical Point):} This is the magic moment. As the average degree hits exactly one, the small islands begin to connect. Suddenly, a single, connected component emerges that is significantly larger than all the others. This is the birth of the \textbf{giant component}.
    
    \item \textbf{When $\lambda > 1$ (Supercritical Phase):} Once the average degree surpasses one, the giant component grows rapidly, absorbing the smaller islands and a large fraction of any newly added nodes. The network is now a cohesive whole.
\end{itemize}

Within this giant component, the average path length is remarkably short, scaling with the logarithm of the network size, $\log(N)$. The random nature of the connections, while not creating hubs, ensures that there are always enough long-distance shortcuts to prevent the ``40-million-step'' problem seen in purely ordered networks.

To visualize this dramatic phase transition, we generated random networks of fixed size $N = 250$ at five strategic values of $\lambda$ around the critical point. For each value, we computed the corresponding edge probability $p = \lambda / (N - 1)$ and constructed a $G(N, p)$ graph. The results are shown in Figure~\ref{fig:phase_transition}, where nodes belonging to the largest connected component (the ``giant component'') are colored red, while all other nodes are gray.

\begin{figure}[htbp]
    \centering
    \begin{subfigure}[b]{0.32\textwidth}
        \includegraphics[width=\textwidth]{figures/phase_transition_lambda_0.5.png}
        \caption{$\lambda = 0.5$ (Subcritical)}
        \label{fig:lambda_0.5}
    \end{subfigure}
    \hfill
    \begin{subfigure}[b]{0.32\textwidth}
        \includegraphics[width=\textwidth]{figures/phase_transition_lambda_0.8.png}
        \caption{$\lambda = 0.8$ (Subcritical)}
        \label{fig:lambda_0.8}
    \end{subfigure}
    \hfill
    \begin{subfigure}[b]{0.32\textwidth}
        \includegraphics[width=\textwidth]{figures/phase_transition_lambda_1.0.png}
        \caption{$\lambda = 1.0$ (Critical)}
        \label{fig:lambda_1.0}
    \end{subfigure}
    
    \vspace{0.5cm}
    
    \begin{subfigure}[b]{0.32\textwidth}
        \includegraphics[width=\textwidth]{figures/phase_transition_lambda_1.2.png}
        \caption{$\lambda = 1.2$ (Supercritical)}
        \label{fig:lambda_1.2}
    \end{subfigure}
    \hfill
    \begin{subfigure}[b]{0.32\textwidth}
        \includegraphics[width=\textwidth]{figures/phase_transition_lambda_2.0.png}
        \caption{$\lambda = 2.0$ (Supercritical)}
        \label{fig:lambda_2.0}
    \end{subfigure}
    
    \caption{The phase transition in Erdős-Rényi random networks. Each panel shows a network of $N = 250$ nodes at different average degrees $\lambda$. Red nodes belong to the largest connected component (giant component), while gray nodes belong to smaller isolated clusters. (a) At $\lambda = 0.5$, the network is a disconnected archipelago of tiny fragments. (b) As $\lambda$ approaches 1, components grow but remain fragmented. (c) At the critical point $\lambda = 1.0$, a giant component suddenly emerges, connecting a significant fraction of nodes. (d--e) In the supercritical phase ($\lambda > 1$), the giant component dominates, creating a cohesive, fully connected network.}
    \label{fig:phase_transition}
\end{figure}

The visualization dramatically demonstrates the theoretical predictions. In the subcritical regime (Figures~\ref{fig:lambda_0.5}--\ref{fig:lambda_0.8}), the network consists of many small, disconnected clusters with no dominant structure. At the critical point (Figure~\ref{fig:lambda_1.0}), we witness the sudden emergence of a large connected component that spans the network. In the supercritical phase (Figures~\ref{fig:lambda_1.2}--\ref{fig:lambda_2.0}), the giant component rapidly absorbs nearly all nodes, creating a single cohesive network capable of supporting widespread cascades.

\subsubsection{Implications for Our Contagion Simulation}

Based on this deep dive, we can formulate a clear set of hypotheses for how an epidemic will behave in a random network:

\begin{enumerate}
    \item \textbf{No Super-Spreader Events}: Because the degree distribution is tightly centered around the average and there are no hubs, we predict that no single node's infection will be catastrophically more impactful than any other's. The spread should be relatively uniform.
    
    \item \textbf{Potential for Widespread but Not Explosive Growth}: The short average path length means the virus has the potential to reach most of the network. However, the lack of hubs and low clustering means it cannot spread with the explosive, exponential velocity that a super-spreader event would cause.
    
    \item \textbf{Predictable Epidemic Curve}: We hypothesize that the epidemic curve (number of infected over time) will follow a classic, relatively symmetric bell shape. The growth will be steady and predictable, lacking the sharp, unpredictable peaks that might be caused by more complex network structures.
\end{enumerate}

This model, in its elegant simplicity, provides the perfect canvas upon which to paint our first simulation, giving us the essential baseline we need to appreciate the profound effects of network structure that we will explore next.

\end{document}